% Intro section
\section{Introduction}

The report documents the project of subtask 5 in the course RSD, Robot System Design. The subtask is one of many, developing a laundry service for home care of elders. \\ This report is written for instructor and individual with knowledge of the subject, the technology, the research and the task at hand.

\subsection{Project Description}

The the beginning of the course we were handed a description of subtask 5:

\begin{itemize}
\item Task description
	\begin{itemize}
	\item Everybody knows that washing black and white together is a bad idea, but there are a lot more criteria to consider, to prepare a wash
		\begin{itemize}
		\item Appropriate colors for joint was
		\item Washing conditions - e.g. temperature, centrifugation, spinning speed, drying
		\end{itemize}
	\item The appropriate amount of laundry should be prepared in piles or containers, which correspond to the load size of a washing machine
	\item When elders move in their clothes will be tagged (subtask 6), but the washing recipes of each item should also be registered
		\begin{itemize}
		\item That task should be simple for the staff
		\end{itemize}
	\end{itemize}
\item Challenges
	\begin{itemize}
	\item Together with group of subtask 6 create an easy an appropriate interface for registering clothes and washing recipes
	\item Control criteria for the robot arm in subtask 4
	\item Close collaboration with subtask 4 and 6
	\item Perhaps some “untagged” clothes should be put in washing bags?
	\end{itemize}
\item Technologies available
	\begin{itemize}
	\item ???
	\end{itemize}
\end{itemize}

The description explains the task, the challenges the task poses and that we should investigate the technologies available. The project collaborates with many of the other subtask, as described in the following sections.