%r
\section{Review of Laundry Facilities}
Washing machines are used in traditional and industrial facilities. Differences of those two areas are water usage, clothes amount and the amount of machines to handle the work load. On table \ref{tab:AdDis} we illustrate the differences.

\begin{table}[h]
	\centering
    \begin{tabular}{ | p{3.5cm} | p{3.5cm} | p{3.5cm} | p{3.5cm} |}
    \hline
    \multicolumn{2}{|c|}{\textbf{Commercial}} & \multicolumn{2}{|c|}{\textbf{Non Commercial}} \\ \hline
    \multicolumn{2}{|l|}{Hotels, hospitals, rest homes and etc.} & \multicolumn{2}{|l|}{Rooms, dormitories and etc.} \\ \hline
    Advantages & Disadvantages & Advantages & Disadvantages \\ \hline
    High efficiency. Usage of the water can be decreased. (High amounts of clothes can be washed more efficiency than small amounts. For instance during one wash, washing machine utilizes the same amount of water. Higher amounts of clothes can be more optimally to wash then small. & Delays. (1. Maybe sometimes one need to wait while laundry room wash your clothes. 2. Person needs to travel from home to laundry room to wash his clothes. Both situations uses a persons time) & Cheaper, faster. (It is sometimes faster to wash at home than to move somewhere in the city) & High usage of water. (The usage of water is not optimal: During one wash it is not always possible to find the max amount of clothes that would be optimal for the washing machine.) \\ \hline
    \end{tabular}
	\caption{Advantages/disadvantages of }
	\label{tab:AdDis}
\end{table}