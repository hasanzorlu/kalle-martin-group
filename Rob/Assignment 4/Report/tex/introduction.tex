\section{Introduction}
\label{sec:intro}

This assignment is apart of the Rob1 course. As a starting point this assignment is supposed to give the students experience of developing a small application that make use of the topics learned so far in the course, such as making plugins for RobWork, using the RobWork interface and additionally to that the students should interface the application to a physical robot and using a tool that perform some interaction with the surroundings.
\\The assignment given to this group of student will make use of a Fanuc LR Mate 200i that is setup to physically carve in styrofoam with a milling tool mounted on the sixth joint of the Fanuc.
\\The user of the robot should be able to type in a sentence in a GUI with a given font and size. From this GUI a task description that tells the robot the lines it has to follow to mill the letters in the foam, is created. The task description must then be translated into joint coordinates that will be uploaded to the Fanuc robot.
\\But before the joint coordinates is uploaded to the robot it must be simulated in RobWorkStudio to make sure that it doesn't damage itself, the milling tool or the surroundings. Some thoughts of how the the tool should be moved when it's airborne and when it is docked to the foam and maby how and where to start and stop the milling tool. Another part of the assignment is to specify the work area where the foam is placed.